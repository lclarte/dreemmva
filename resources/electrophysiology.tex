\section{Electrophysiology of sleep}

Whilst sleep is commonly viewed as an homogeneous phenomenon through which restorative tasks are being carried by our organism, EEG recordings of normal subjects show that sleep actually comprised different stages which correspond to different electrophysiological regimes. In what follows, we begin with a rough introduction of those different regimes by presenting some of their defining characteristics. Doing so pertains to us as a paramount step toward building interpretable predictive models, as this allows for a better understanding of the underlying geometry to the sheer data.  We then mention some current literature discussions on gender-based differences of sleep EEG, so as to make our way into assembling a first evidence-based predictive model.
\\
\paragraph{Sleep cycle.} The sleep cycle divides into two defining regimes, namely the \emph{Rapid Eye Movement} sleep (REM) and the non-REM sleep. The REM sleep regime is distinguishable, as hinted by its very denomination, by the presence of erratic and rapid movements to the subject's eyes. In REM sleep, the EEG recordings are very similar to that of the awake state (\emph{i.e.} low-voltage and high-frequency electrophysiological activity), to the extent that REM sleep also bears the name of \emph{paradoxal sleep}. The non-REM sleep phase is divided into four stages, namely:

\begin{itemize}
\item \emph{Stage I}. This stage is characterized by brainwaves of medium frequencies ($4-8$ Hz) with increasing amplitudes ($50 - 100$ $\mu$V) compared to the waking state, which is characterized by high-frequency ($15-60$ Hz) and low amplitude ($\sim 30$ $\mu$V) activity, the so-called \emph{beta waves}.

\item \emph{Stage II}. This stage is characterized by the presence of \emph{sleep spindles}, which are sudden burst of oscillatory brain activity, corresponding to frequencies bandwidth $10-15$ Hz and amplitudes $50-150$ $\mu$V. Research indicates that those sleep spindles stem from the interactions between thalamic and cortical neurons. Another type of defining brainwaves for the stage II non-REM phase are the so-called \emph{K-complex}, which, unlike the sleep spindles, are waves of low frequencies ($0.5 - 4$ Hz) and large amplitude that react to external stimuli while sleeping. 

\item \emph{Stage III}.  This stage, which represents moderate to deep sleep, is characterized by slower waves at $2 - 4$ Hz with amplitudes $100 - 150$ $\mu$V. 

\item \emph{Stage IV}. In this final stage of the non-REM sleep phase, which represents the deepest level of sleep, the predominant EEG activity displays brainwaves of low frequency ($1-4$ Hz) and high-amplitudes, the so-called \emph{delta waves}.
\end{itemize}

After reaching Stage IV, the sequence reverses itself and a period of REM sleep ensues. This pattern is repeated 
\paragraph{Toward gender-based differences of sleep physiology}