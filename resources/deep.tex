\section{Deep learning-based classification}

% Resumer le resultat obtenu 
In this section, we develop the deep learning pipeline used for the gender classification. 

The original article on which the challenge is based proposed \cite{vanPutten2018} describes an architecture based on a previous architecture designed by Krizhevsky et al. to work on ImageNet \cite{Krizhevsky2012}. 

% TODO : reproduire l'architecture, sinon dire qu'on a essayé de la reproduire et que ca n'a pas marché

\subsection{Reproducing the article's network}

The first of this project was to reproduce the results presented in\cite{vanPutten2018}. 

\paragraph{Data formating} The authors of\cite{vanPutten2018} studied EEG recordings of 2 seconds with a sampling frequency of 128 Hz and 24 channels, thus having data formatted as 24 x 256 matrices. Since our input data has the shape of 7 x 500 matrices, we decided to resize them to the network's input shape.
% TODO : Dire que ça n'a pas de signification physique mais que c'est motivé ar le fait que l'architecture est inspirée d'ImageNet (donc on fait comme avec le traitement d'image).

For hyperparameter tuning, we used the python library \texttt{keras-tuner}

\subsection{Braindecode library}

In addition, we used the Python library \cite{braindecode}, which is specialized in classification of EEG recordings using deep-learning and built on top of the Pytorch library.

\paragraph{Deep neural network}

\paragraph{Visualizing the network parameters}
