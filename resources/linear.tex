\section{Classification with linear models}

\paragraph{Linear classifier using DFT. } Recall that a signal $x(t) \in L^2(\mathbb{R})$ is entirely characterized by its Fourier transform $\widehat{x}(\omega) \in L^2(\mathbb{R})$ and vice-versa (\emph{i.e.} $x(t) \mapsto \widehat{x}(\omega)$ is a bijective isometry of $L^2(\mathbb{R})$). Therefore, by working on the frequency-signal $\widehat{x}(\omega)$ rather that on the time-signal $x(t)$, we do not lose (\emph{a priori}) any relevant information concerning the recorded time-signal. Because the EEG data at our disposition is not time-stamped, the temporal structure is so to say inexistant, which advocates for carrying the computations into the frequency-domain. Given a discrete time-signal $\{x_0, \dots, x_{N-1} \}$, the \emph{Discrete Fourier Transform} (DFT) is defined as the frequency-signal $\{\omega_0, \dots, \omega_{N-1}\}$, where
$$
\omega_k := \sum_{j = 0}^{N-1} x_j \exp( - i 2 \pi kj/N), \quad k = 0, \dots, N-1. 
$$


Because the recordings are our disposal each only spans for a short duration $T$ across time (\emph{i.e.} $T = 2$ sec), there exists a lower-bound $\omega_{low}$ in the frequency-domain below which the Fourier transform is simply unable to perform anymore, that is $\omega_{low} = 1/T$ (\emph{i.e.} $0.5$ Hz in our context). This is a well-known drawback to the short-time Fourier transform. Nevertheless, because of the nature of our data as presented above, we shall not be expecting such low-frequency features in the EEG signal.

Inherent to any dataset obtained through physical-acquisition is the presence of measurement noise. For each EEG channel, we want to estimate the \emph{power spectral density} (PSD) (\emph{i.e.} the quantity $| \widehat{x}(\omega)|^2$) as accurately as possible. One way to proceed is the \emph{Bartlett's method}, which consists in splitting the recorded signal $\{x_0, \dots, x_{KQ - 1}$ of size $KQ$ into $K$ segments disjoint of size $Q$, \emph{i.e.} $(\{x_{k(Q - 1)}, \dots, x_{(k+1)Q - 1}\})_{k=1}^K$, to compute the spectral density for each segment through a certain weight function (\emph{i.e.} the window) and to average over the segments. This allows for a smaller variance in the measurement (\emph{i.e.} less measurement noise). An improvement of the Bartlett's method is the so-called \emph{Welch's method} which introduces a overlap of user-defined size $D$ between the different segments $s_k$'s (\emph{i.e.} $|s_k \cap s_{k+1}| = D$). Since the windows are generically more sensitive to the data at the centers of the segments, the Bartlett's method results in a loss of information, which here is alleviated by the overlapping. 


Using Welch's method, we compute the PSD of men and women over different splits of the training set (we average over the $40$ samples for each subject). First, \emph{(i)} we observe trends for each channel that are quite similar from split to split, which indicates a common underlying geometry to the data (\emph{i.e.} the data is consistent across all subjects). Over some splits, \emph{(ii)} we observe a mild but nevertheless significant difference between genders. Nevertheless, \emph{(iii)} some splits yield very similar results for both male and female, which therefore tends to indicate that the variance of the Welch estimator is rather to be sought in inter-subject variability than in inter-sex variability.


Nevertheless, we perform a simple classification 
\paragraph{Linear classifier using DWT}