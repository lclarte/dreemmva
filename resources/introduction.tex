\section{Introduction}

\paragraph{Presentation}The goal of this data challenge is to predict a subject's gender (\emph{i.e.} male or female) from the electrical activity of its brain during sleep.
The dataset provided by Dreem for this challenge is divided into a \textit{training set} and a \textit{test set}. The training set and the test set both contains EEG measurement for $946$ subjects. On each of these subjects, the EEG signal has been recorded on $7$ channels on $40$ time intervals of $2$ seconds. The sampling frequency of the signal is $250$ Hz. As a consequence, each subject is associated with an input array of size $(40, 7, 500)$.
 
\paragraph{Notation} Hereafter, male subjects will be denoted as << class $0$ >> and female subjects will be assigned << class $1$ >>.

\paragraph{Previous work}

The challenge webpage was presented with a reference article \cite{nature_original}. In this work, the author applied a deep-learning model to the prediction of the gender from EEG recordings. Using a convolutional neural network, the authors achieved 81\% of accuracy. 

\paragraph{Data balancing} The repartition of genders in the training set is 737 men and 209 women, i.e male subjects represent 78\% of the training set. As a consequence, a trivial classifier returning 0 will have 78\% accuracy on the training step. It is natural to ask whether or not the test set is also imbalanced. Since, we do not have access to the true classes of the test set, we simply submitted a solution on the challenge webpage where all the predictions where male, and obtained and accuracy of around 77.5\%. Thus the test set has this 77.5\% of male subjects and is quite imbalanced like the training set. \\
% Trouver des references
We experimented two different strategies to solve the problem of class imbalance : first, we assigned different weights to each sample, following the approach described in \cite{King2001}. On the other hand, we tried undersampling the majority class (i.e men) to have the same amounf of men and women in our dataset. The drawback of this approach is that we reduce the size of an already small dataset. Indeed, since there are 209 women in the original file x\_train, the dataset after undersampling includes 209 men and 209 women i.e 418 subjects compared to 946 initially. 